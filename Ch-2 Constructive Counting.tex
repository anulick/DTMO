\chapter{Permutations and Combinations and a pinch of Probability}
\section{Winning the lottery}
As you are reading this book on math, you might be interested in math. Maybe one of your relatives have asked you what their chance of winning the lottery is. Here you can simply answer, "Your chances of winning the lottery are $1$ in $13,983,816$" which should do the trick unless one day they ask you "why?"\\
Here your answer will be: "Because there are $49$ numbers to choose from and you can choose any $6$ of them. Therefore the total combinations comes to be $\frac{49!}{6!*(49-6)!}$ which comes out to be $13,983,816$; of which only one is the winner." \\
This will settle them. But you will want to know from where did $\frac{49!}{6!*(49-6)!}$ come and what does it mean?\\
\section{The special sign!}
Normally when you see a ! sign in a book, it refers to the author trying to be funny. But if it comes after a number it means as follows:\\
\begin{definition}
    [Factorial]
    $n!=n*(n-1)*(n-2) \dots 3*2*1$\\
\end{definition}
This comes in use in a variety of places, including in Burger King. Let's say you meet me outside and we start talking about math. We feel hungry so I treat you to a burger king meal. You have a burger, a fries and a coke. In how many ways could you consume the three of them, eating one at a time.\\
You have three choices to start your meal of with. Once you consume that, you have two choices. For example: If you ate the burger first, you can now have the fries or drink the coke. Finally, you are left with only one consumable and hence one choice. So in total you can have the meal in $3*2*1=6$ ways.\\
This can be generalized as: 
\begin{theorem}
    The number of ways to arrange $n$ objects is $n!$\\
\end{theorem}
We will also can see that:
\begin{theorem}
The number of ways of arranging $n$ objects in a circle where rotations of the same arrangement are not considered distinct is $(n - 1)!$ \\
\end{theorem}
This is obvious when we notice that only $1$ in $n$ arrangements must be counted as the rest become same when arranged in a circle.  Similerly:
\begin{theorem}
The number of ways of arranging $n$ objects in a circle where rotations of the same arrangement are not considered distinct and reflections of the same arrangement are not considered distinct is $\frac{(n-1)!}{2}$
\end{theorem}
\section{Criminal Lineups}
If we have $5$ suspects lined up, we can arrange them in $5!=120$ ways. But if 2 of them are wearing squid game masks them? Now they are identical and hence interchangeable. The ways to arrange them now will be halved as they both being switched doesn't create a new permutation.\\
Now what if the reaming three of them wear Joker masks. We'll have to divide the permutations by $3!=6$ as all of them are identical.  Generalizing:
\begin{theorem}
    The number of ways to arrange $a$ total objects if $k, l, m, n \dots$ are identical = $\frac{a!}{k!l!m!n! \dots}$
\end{theorem}
Unfortunately,  no such simple formula/theorem exists for circular permutations. They need to be solved on case by case basis.
\begin{example}
    Using the letters $M, O, P, R,$ and $x$, we can form five-letter ”words”. If these ”words” are arranged in alphabetical order, then what position does the ”word” $PRMOx$ occupy?
\end{example}
\begin{proof}
    [Solution]
    We can first look at all words starting with letter $M$ which is alphbetically the first.\\
    There are $4!$ such words. Similarly for $O$ brigs the total to $48$\\
    In words staring with $P$ we first look at words which start with $PM$ which we have $3!$ of. Then $PO$.\\
    This brings the total count to $48+12=60$\\
    $PRMOx$ is the $61$st word as after $PR$ the rest of the letters are in alphabeticaly order and hence the very next word.\\
\end{proof}
\section{Cricket team selection}
Your school is having an inter-class cricket tournament. From every class of $30$ we need to choose $11$ players. How many ways can we do it?\\
For the first players we have $30$ choices, then $29$ and so on. But that's not all. The order in which players are chosen doesn't matter as they a are a team in the end. So we need to divide it in the end by $11!$. So the number of possible teams will be $\frac{30*29*28 \dots 22*21 *20}{11!}=54,627,300$. Generalizing:\\
\begin{theorem}
    Number of ways of choosing k objects from n, where order doesn't matter is $\binom{n}{k}=\frac{n!}{k!*(n-k)!}$
\end{theorem}
\begin{theorem}
    [Remark]
    Notice that $\binom{n}{k}=\binom{n}{k-n}$ as ways of choosing k things to be selected is the same as choosing $n-k$ things to not be selected.
\end{theorem}

\section{Subsets}
A set is a collection of things. A subset is a smaller collection of things all of which are part of the set it is subset of.  \\
\begin{example}
    If a set has $n$ distinct elements in it, How many subsets of that set exist?
\end{example}
\begin{proof}
    [Solution]
    Every element is either in the subset or not in it. Hence we have two possibilities for every element. Hence we can say $2^n$ subsets exist.
\end{proof}
\begin{theorem}
[Subset Theorem]
    The number of subsets of a set of size n is $2^n$.
\end{theorem}
Note that we have considered the empty set(the one with zero elements) to be a subset of the set. Please check if the question is considering the same. If not subtract $1$ from $2^n$.
Also note that no formula exists for set with some repeating elements. We'll solve it using beggars theorem(aka Stars and Bars),  you'll learn more about it later.
\begin{example}
    (AMC 10 2008) Two subsets of the set $S = {a, b, c, d, e}$ are to be chosen so that their union is S and their intersection contains exactly two elements. In how many ways can this be done, assuming that the order in which the subsets are chosen does not matter?
\end{example}
\begin{proof}
    [Solution]
    Let the subsets be $A$ and $B$, hence $A \cup B = S$.\\
    We are basically looking to divide $S$ into three sets. The elements which only lie in $A$, the elements which only lie in $B$ and the elements which lie in $A \cap B$\\
    As $2$ elements lie in $A \cap B$, we have $\binom{5}{2}$ ways to detemine them.\\
    The other three elements need to divided to $A$ and $B$, therefore by using the subset theorem we have $2^3$ ways to do so.\\
    Thus, the total ways to do so are $\frac{2^3 \cdot \binom{5}{2}}{2}=40$. We are dividing by $2$ as we have over counted the case where $A$ and $B$ have just interchanged.
\end{proof}
\section{Probability}
Probability is basically the chance of something occurring. While probability and its theories are their own branch of mathematics, most of it is much closer to statistics than to maths. \\
With a mathematical perspective, The only thing we need to know(which you probably already do) is
\begin{theorem}
    $\text{Probablity} = \frac{\text{Number of desired outcomes}}{\text{Total number of outcomes}}$
\end{theorem}
\begin{example}
    (AIME 2000) A deck of forty cards consists of four $1$’s, four $2$’s,..., and four $10$’s. A matching pair (two cards with the same number) is removed from the deck. Given that these cards are not returned to the deck, let $\frac{m}{n}$ be the probability that two randomly selected cards also form a pair, where $m$ and $n$ are relatively prime positive integers. Find $m + n$
\end{example}
\begin{proof}
    [Solution]
    Without loss of generality, Let the removed pair be of $1$.\\
    The first card can either be $1$ or not be $1$ (obviously). Based on this the probability second card forming a pair is either $\frac{1}{37}$ or $\frac{3}{37}$.\\
    The probability of first card being $1$ is $\frac{2}{38}$ and that of it not being $1$ is obviously $\frac{36}{38}$\\
    Hence, the probability of getting a pair is $\frac{2}{38}\cdot\frac{1}{37}+\frac{36}{38}*\frac{3}{37}=\frac{2+108}{38*37}=\frac{55}{19*37}=\frac{55}{703}$\\
    Thus, $m+n=55+703=758$
\end{proof}
Solve at least questions worth \points{52}. This exercise has a total of \points{68}.
\begin{xcb}{Exercises}
\begin{enumerate}
\item(AMC 10 2019) \points{2} A child builds towers using identically shaped cubes of different colors. How many different towers with a height of $8$ cubes can the child build with $2$ red cubes, $3$ blue cubes, and $4$ green cubes? (One cube will be left out.)?
\begin{hint}
    \addhint {Just build a $9$ cube tower and ignore the last block.}
\end{hint}
\item (AMC 10 2006) \points{2} A license plate in a certain state consists of 4 digits, not necessarily distinct, and 2 letters, also not necessarily distinct. These six characters may appear in any order, except that the two letters must appear next to each other. How many distinct license plates are possible?
\begin{hint}
    \addhint {First choose the numbers and letters and then simply permute them.}
\end{hint}
\item (AMC 10 2017) \points{2} At a gathering of 30 people, there are 20 people who all know each other and 10 people who know no one. People who know each other hug, and people who do not know each other shake hands. How many handshakes occur within the group?
\item (AMC 10 2004) \points{2} Henry’s Hamburger Haven serves its hamburgers with the following condiments: ketchup, mustard, mayonnaise, tomato, lettuce, pickles, cheese, and onions. A customer can choose one, two, or three meat patties and any collection of condiments. How many different kinds of hamburgers can be ordered?
\item (AMC 12 2022) \points{5} What is the number of ways the numbers from 1 to 14 can be split into 7 pairs such that for each pair, the greater number is at least 2 times the smaller number?
\begin{hint}
    \addhint {The integers from $8$ through $14$ must be in different pairs, and $7$ must pair with $14.$. Why is this true? And why does this solve this question?}
\end{hint}
\item(AMC 12 2003) \points{5} How many 15-letter arrangements of 5 A’s, 5 B’s, and 5 C’s have no A’s in the first 5 letters, no B’s in the next 5 letters, and no C’s in the last 5 letters?
\begin{hint}
    \addhint {If we have $x$ B's in the first $5$ letters and $5-x$ C's, what will happen? Does this solve the question?}
\end{hint}
\item(AMC 10 2021) \points{2} A deck of cards has only red cards and black cards. The probability of a randomly chosen card being red is $1/3$ . When 4 black cards are added to the deck, the probability of choosing red becomes $1/4$ . How many cards were in the deck originally?
\item(AMC 10 2006) \points{2} Bob and Alice each have a bag that contains one ball of each of the colors blue, green, orange, red, and violet. Alice randomly selects one ball from her bag and puts it into Bob’s bag. Bob then randomly selects one ball from his bag and puts it into Alice’s bag. What is the probability that after this process the contents of the two bags are the same?
\item(AMC 10 2020) \points{5} Ms. Carr asks her students to read any 5 of the 10 books on a reading list. Harold randomly selects 5 books from this list, and Betty does the same. What is the probability that there are exactly 2 books that they both select?
\begin{hint}
    \addhint {Think about why we really don't care which books Harold selects.}
\end{hint}
\item (AMC 12 2021) \points{9} Two fair dice, each with at least 6 faces are rolled. On each face of each dice is printed a distinct integer from 1 to the number of faces on that die, inclusive. The probability of rolling a sum of 7 is $3/4$ of the probability of rolling a sum of 10, and the probability of rolling a sum of 12 is $1/12$ . What is the least possible number of faces on the two dice combined?
\begin{hint}
    \addhint {In how many ways can we add up to $7$? What does that tell us about ways to add up to $10$? What does that tell us about one of the dice faces?}
    \addhint {Let $n$ be the ways to get $12$ and then try solving the equation.}
\end{hint}
\item (AMC 10 2009) \points{3} Two cubical dice each have removable numbers 1 through 6. The twelve numbers on the two dice are removed, put into a bag, then drawn one at a time and randomly reattached to the faces of the cubes, one number to each face. The dice are then rolled and the numbers on the two top faces are added. What is the probability that the sum is 7?
\begin{hint}
    \addhint {Is rolling in anyway different from just pulling two numbers from the bag?}
\end{hint}
\item (AMC 12 2019) \points{3} The numbers $1,2 \dots ,9$ are randomly placed into the 9 squares of a 3 by 3 grid. Each square gets one number, and each of the numbers is used once. What is the probability that the sum of the numbers in each row and each column is odd?\\
\begin{hint}
    \addhint {In what ways can we get an odd sum? Can you arrange $E$ and $O$ in such a way that this is true? How many rearangments does your arrangement have?}
\end{hint}
\item (AMC 12 2003) \points{2} Let $S$ be the set of permutations of the sequence $1, 2, 3, 4, 5$ for which the first term is not $1$. A permutation is chosen randomly from $S$. The probability that the second term is $2$, in lowest terms, is $a/b$. What is $a + b$?
\item (AMC 10 2018) \points{2} A box contains $5$ chips, numbered $1, 2, 3, 4,$ and $5$. Chips are drawn randomly one at a time without replacement until the sum of the values drawn exceeds $4$. What is the probability that $3$ draws are required?
\begin{hint}
    \addhint {Just write all the possible draws till we exceed $4$ and you'll be done in no time.}
\end{hint}
\item (AMC 10 2021) \points{3} Each of the $20$ balls is tossed independently and at random into one of the $5$ bins. Let $p$ be the probability that some bin ends up with $3$ balls, another with $5$ balls, and the other three with $4$ balls each. Let $q$ be the probability that every bin ends up with 4 balls. What is $p/q$ ?
\begin{hint}
    \addhint {Assume that the balls and bins are both distinguishable and then this is just question 2, but a bit more involved.}
\end{hint}
\item (ISRO Interview) \points{3} A bag contains $2007$ red balls and $2007$ black balls. We remove two balls
at a time repeatedly and\\
(i) discard them if they are of the same color,\\
(ii) discard the black ball and return to the bag the red ball if they are of different
colors.\\
What is the probability that this process will terminate with one red ball in the bag?
\begin{hint}
    \addhint {We can draw BB, BR, RB or RR. What happens in each case?}
\end{hint}
\item \points{5} Two evenly matched teams play in the world series, a best of seven competition in which the competition stops as soon as one team has won four games. Is the world series more likely to end in six or seven games?
\item You toss $n$ coins, and you win if you turn up an even number of heads. Otherwise, Bob Hough takes your lunch money.\\
(a) \points{5} Show that your odds of winning are $50\%$ if all the coins are fair coins.\\
(b) \points{3} Better yet, show that your odds of winning are $50\%$ if at least one of the coins is
fair.\\
\item(AMC 10 2018) \points{3} Three young brother-sister pairs from different families need to take a trip in a van. These six children will occupy the second and third rows in the van, each of which has three seats. To avoid disruptions, siblings may not sit right next to each other in the same row, and no child may sit directly in front of his or her sibling. How many seating arrangements are possible for this trip?
\end{enumerate}
\end{xcb}
